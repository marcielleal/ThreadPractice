\documentclass[12pt]{article}
\usepackage[english,portuguese]{babel}
\usepackage{natbib}
\usepackage{url}
\usepackage[utf8x]{inputenc}
\usepackage{amsmath}
\usepackage{graphicx}
\graphicspath{{images/}}
\usepackage{parskip}
\usepackage{fancyhdr}
\usepackage{vmargin}
\usepackage{caption}
\usepackage{hyperref}
\usepackage{float}
%%%%%%%%%%%%%%%%%%%

\usepackage{graphicx}
\usepackage{caption}
\usepackage{subcaption}
\usepackage{placeins}
\usepackage{fancyvrb}
\usepackage{mathtools}

\newcommand\tab[1][1cm]{\hspace*{#1}}
%%%%%%%%%%%%%%%%%%%

\setmarginsrb{3 cm}{2.5 cm}{3 cm}{2.5 cm}{1 cm}{1.5 cm}{1 cm}{1.5 cm}

\title{Trabalho prático: Programação com threads}								% Title
\author{Marciel Manoel Leal}	% Author
\date{\today}											% Date

\makeatletter
\let\thetitle\@title
\let\theauthor\@author
\let\thedate\@date
\makeatother

\pagestyle{fancy}
\fancyhf{}
\rhead{\theauthor}
\lhead{\thetitle}
\cfoot{\thepage}

\begin{document}

%%%%%%%%%%%%%%%%%%%%%%%%%%%%%%%%%%%%%%%%%%%%%%%%%%%%%%%%%%%%%%%%%%%%%%%%%%%%%%%%%%%%%%%%%

\begin{titlepage}
	\centering
    \vspace*{0.5 cm}
    \includegraphics[scale = 2]{ufrn.jpg}\\[1.0 cm]	% University Logo
    \textsc{\vspace{0.5 cm} \LARGE Universidade Federal do Rio Grande do Norte}\\[2.0 cm]	% University Name
	\textsc{\Large DIM0612}\\[0.5 cm]				% Course Code
	\textsc{\large Programação Concorrente}\\[0.5 cm]				% Course Name
	\rule{\linewidth}{0.2 mm} \\[0.4 cm]
	{ \huge \bfseries \thetitle}\\
	\rule{\linewidth}{0.2 mm} \\[1.5 cm]
	
	\begin{minipage}{0.4\textwidth}
		\begin{flushleft} \large
			\emph{Discentes:}\\
			\theauthor
			\end{flushleft}
			\end{minipage}~
			\begin{minipage}{0.4\textwidth}
			\begin{flushright} \large
			\emph{Matrícula:} \\
			2015039785								% Your Student Number
		\end{flushright}
	\end{minipage}\\[2 cm]
	
	{\large \today}\\[2 cm]
 
	\vfill
	
\end{titlepage}

%%%%%%%%%%%%%%%%%%%%%%%%%%%%%%%%%%%%%%%%%%%%%%%%%%%%%%%%%%%%%%%%%%%%%%%%%%%%%%%%%%%%%%%%%

\tableofcontents
\pagebreak

%%%%%%%%%%%%%%%%%%%%%%%%%%%%%%%%%%%%%%%%%%%%%%%%%%%%%%%%%%%%%%%%%%%%%%%%%%%%%%%%%%%%%%%%%
\section{Introdução}
Tendo em vista que muitos processadores modernos funcionam com mais de um núcleo, muitas vezes os programadores fazem softwares que subutilizam os recursos computacionais disponíveis, pois não possuem muitos conhecimentos em programação concorrente ou não querem  fazer uso dela \textbf{TODO}. No entanto a programação concorrente pode gerar muitos bugs inesperados e muitas vezes ir-rastreáveis, além disso, ela pode não ter um ganho tão significativo, dependendo do contexto.

Este documento visa relatar um pequeno experimento sobre programação concorrente. O objetivo desse experimento é comparar a eficiência da programação concorrente com a programação sequencial no problema de multiplicação de matrizes.

\section{Detalhes de implementação}
A implementação auxiliar se divide em dois pacotes descritos a seguir:
\begin{itemize}
\item \textbf{squarematrix.py} possui a implementação da classe \textbf{SquareMatrix} que encapsula a criação de uma matriz quadrada e o acesso a seus elementos
\item \textbf{multiplier.py} possui a implementação de algumas funções para a multiplicação de matrizes:
	\begin{itemize}
	\item \textbf{matrixmult} é a função chamada na multiplicação de matrizes. Ela possui os parâmetros A e B, que são as matrizes a serem multiplicadas e possui um parâmetro opcional que é o número de threads, caso ele seja passado e esse número for maior que 1, a função fará a multiplicação utilizando multiplas threads, caso contrário, fará de forma sequencial.
    \item Seguindo as especificações do \href{https://www.python.org/dev/peps/pep-0008/}{PEP8}, as seguintes funções são funções internas do pacote, que auxiliam na execução da função \textbf{matrixmult}:
    \begin{itemize}
    \item \textbf{\_\_rowmultmatrix} multiplica uma linha de uma matriz quadrada A pelas colunas de B (é, digamos, o núcleo da multiplicação);
    \item \textbf{\_runthread} é uma função target para a criação das Threads.
    \end{itemize}
	\end{itemize}
\end{itemize}

Há mais dois arquivos que são os scripts de fato. Esses scripts recebem um tamanho de matriz como parâmetro, caso esse tamanho não for de acordo com os requisitos, o programa não executa. Os scripts são descritos abaixo:
\begin{itemize}
\item \textbf{multimat\_sequencial.py} é o script que multiplica matrizes sequencial. É necessário a passagem de um parâmetro para a sua execução. Esse parâmetro é a dimensão da matriz;
\item \textbf{multimat\_concorrente.py} é o script que multiplica matrizes de forma concorrente. É necessário a passagem de dois parâmetros para a sua execução. O primeiro parâmetro é a dimensão da matriz e o segundo é a quantidade máxima de threads que podem ser criadas.
\end{itemize}
O objetivo dos scripts é fazer a multiplicação de duas matrizes: A e B. Essas matrizes estão armazenadas em arquivos nomeados como \textit{tipoDimensaoxDimensao}, onde o tipo pode ser A ou B, e o Dimensao é o valor da dimensão da matriz. Todas essas matrizes estão dentro da pasta \textbf{Matrizes}. Logo, ao ser passada uma dimensão $d$ para um dos scripts, ele irá abrir os arquivos \textbf{Adxd} e \textbf{Bdxd}. Criará dois objetos do tipo \textbf{SquareMatrix} e irá chamar a função \textbf{matrixmult}, passando esses objetos como parâmetros. No caso do script concorrente, ele passará também o número de threads que foi recebido via terminal.

\section{Metodologia}
O experimento utilizou-se de matrizes quadradas. Os requisitos para as dimensões dessas matrizes foram: 
\begin{itemize}
	\item ser uma potência de 2
	\item estar no intervalo estão no intervalo $[4,1024]$
\end{itemize}

O tempo foi medido utilizando a função \href{https://docs.python.org/3/library/time.html#time.time}{time} da biblioteca time. O tempo de cada script foi impresso na saída padrão.

As exceções foram colocadas para serem impressas na saída padrão. Nenhuma exceção foi levantada no experimento definitivo.

Os experimentos foram realizados utilizando um script \textbf{tests.sh}. Esse script possui um loop de 20 iterações, onde chama os dois scripts e redireciona suas saídas para arquivos. As saídas do executável sequencial foram armazenadas na pasta \textbf{MeasuresSeq} e as do concorrente na pasta \textbf{MeasuresConc}. O nome desses arquivos de saída seguem a sintaxe \textit{dimensaoxdimensao}, onde dimensao é a dimensão passada ao script. O número de Threads utilizada para cada execução do script concorrente foi metade da dimensão da matriz, ou seja, cada Thread multiplicou duas linhas da sua respectiva matriz A.

Os dois scripts foram testados para todas as dimensões de matrizes antes do experimento, verificando-se a corretude da matriz-resultado, e, portanto, dos algoritmos. No experimento, a matriz-resultado foi calculada, mas não foi impressa, já que a corretude da saída não era o objetivo dos testes.

Utilizou-se de um computador com as seguintes especificações:
	\begin{itemize}
		\item Memória RAM: 8GB
        \item Sistema operacional: Ubuntu 16.04 LTS 64 bits
        \item Processador: Intel Core i7-5500U CPU @ 2.40GHz × 4
	\end{itemize}

A linguagem utilizada para o script de testes foi Shell Script, executado no bash versão 4.3.48.

A linguagem utilizada foi Python em sua versão \href{https://www.python.org/downloads/release/python-352/}{3.5.2}.

\section{Resultados}
As podem ser vistas nessa \href{https://docs.google.com/spreadsheets/d/14SQn6Wzfrw42DPdRM_UAaic5HaNJHd4XLTSgcDLYDHU/edit?usp=sharing}{planilha} e alguns resultados tirados podem ser lidos abaixo (todos os números estão em segundos):

\begin{table}[H]
\centering
\caption{Script sequencial}
\begin{tabular}{lllll}
 Dimensão&Média &Máximo  &Mínimo  &Desvio Padrão  \\
 4&	0.0002907991409	&  0.0006022453308	& 0.0002374649048 & 0.0001066008816\\
 8&  0.002117300034&  0.02339220047& 0.00088763237 & 0.005010325641 \\
 16&  0.005999982357&  0.007823228836& 0.00562620163 & 0.000496041051 \\
 32&  0.04389901161& 0.0468583107 & 0.04241871834 & 0.00111283432 \\
 64&  0.3491618276&  0.4293854237& 0.3288066387	 & 0.0245424195 \\
 128& 2.660050738&  2.954339743	& 2.540849924	 & 0.1020514268 \\
 256& 20.83647889&  21.44054699	& 20.37756658 & 0.2901211262 \\
 512& 168.2958302&  175.4287961& 163.7993152 & 2.563394119\\
 2014& 1361.676831& 1448.133599 & 1304.051964 & 34.12679676
\end{tabular}
\end{table}

\begin{table}[H]
\centering
\caption{Script sequencial}
\begin{tabular}{lllll}
 Dimensão&Média &Máximo  &Mínimo  &Desvio Padrão  \\
 4&	0.001057624817	&  0.004051923752	& 0.0006701946259 & 0.0007554259476\\
 8&  0.006040561199&  0.08232426643& 0.0017619133 & 0.01795630738 \\
 16&  0.009890294075& 0.04310035706& 0.007635116577 & 0.007829143285\\
 32&  0.0488155961& 0.0677189827 & 0.04609394073 &0.004545745098  \\
 64&  0.3510119915&  0.4072928429 & 0.3347120285& 0.01566897484\\
 128& 2.711036754&  2.787316322	 &2.645894527 &0.04007794359 \\
 256& 24.11169454&  30.74346662	&  21.43333721&  2.155104241\\
 512& 173.1249938&  195.6851428	& 167.7998388& 6.614993082\\
 1024& 1415.55427& 1486.977663 & 1372.838478 & 33.68285217
\end{tabular}
\end{table}


\section{Discussão}
Vê-se que o algoritmo sequencial foi melhor que o concorrente na maioria dos casos. Acredita-se que a complexidade para a criação das Threads venceu a tarefa de multiplicação, como esperado. Esse é um exemplo de que, dependendo da implementação, Threads não necessariamente serão mais eficientes.

\end{document}